\documentclass{article}
\usepackage[utf8]{inputenc}
\usepackage[russian]{babel}

\begin{document}

\section{Метрики}
Задача, которую будет решать планировщик можно описать следующим образом: необходимо распределить $N$ задач по $K$ серверам с $C_i$ ядрами. При этом, обычные планировщики запускают задачи на фиксированном количестве процессоров, а “Планировщик распределения задач вычислительного кластера” сможет варьировать количество процессоров. 
Для сравнения результатов работы запусков двух планировщиков (rigid и moldable) будут использоваться следующие метрики: 

1) общее время выполнение пачки из $N$ задач $$t_{TA}=t_{fin}-t_{start},$$ где $t_{fin}$ - время окончания обработки пачки задач на кластере, $t_{start}$ - время начала обработки пачки задач; 

2) плотность загрузки имеющихся ресурсов $$P =\frac{
\sum_{j=1}^{N} 
t_j c_j}
{t_{TA} 
\sum_{i=1}^{K} 
C_i
},$$ где $t_j$ - время выполнения $j$-ой задачи из пачки,  $c_j$ - количество процессоров задействованных для выполнения $j$-ой задачи из пачки. 

\end{document}
