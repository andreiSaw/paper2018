\documentclass{article}
\usepackage[utf8]{inputenc}
\usepackage[russian]{babel}
\usepackage{amsmath}
\usepackage{amssymb} % for \diagup
\usepackage{mathtools}
\usepackage{tikz} % for tikz solution

\begin{document}

\section{Метрики}
Задача, которую будет решать планировщик можно описать следующим образом: необходимо распределить $N$ задач по $K$ серверам с $C_i$ ядрами. При этом, обычные планировщики запускают задачи на фиксированном количестве процессоров, а “Планировщик распределения задач вычислительного кластера” сможет варьировать количество процессоров. 
Для сравнения результатов работы запусков двух планировщиков (rigid и moldable) будут использоваться следующие метрики: 

1) общее время выполнение пачки из $N$ задач $$t_{TA}=t_{fin}-t_{start},$$ где $t_{fin}$ - время окончания обработки пачки задач на кластере, $t_{start}$ - время начала обработки пачки задач; 

2) плотность загрузки имеющихся ресурсов $$P =\frac{
\sum_{j=1}^{N} 
t_j c_j}
{t_{TA} 
\sum_{i=1}^{K} 
C_i
},$$ где $t_j$ - время выполнения $j$-ой задачи из пачки,  $c_j$ - количество процессоров задействованных для выполнения $j$-ой задачи из пачки. 

\section{Математическая модель}

Пусть $J$ - множество задач, $|J|=N$;
$M$ - множество серверов, $|M|=K$;
$C = \{1,2,..L\}$ - число ядер ($L$ - максимально возможное количество ядер);
\begin{center}
\begin{math}
\forall_m \in M:\exists_{c_i}, i=1,...,L \Rightarrow f:J \nrightarrow M
\end{math}
\end{center}
где $f$ - функция планировщика, а
\begin{center}
\begin{math}
f:J \nrightarrow M \Leftrightarrow
\forall_j \in J : f(j) = \left[ \begin{matrix} m \in M \\ undefined   \end{matrix} \right.,
\end{math}
\end{center}
- частичное отображение.

Таким образом, частичное отображение подразумевает работу планировщика в формате: 1) если есть подходящие ресурсы - разместить на ресурсе; 2) если подходящих ресурсов на данный момент нет - задача не размещается. 

Качество решения тем лучше, чем меньше общее время выполнения пачки задач, и выше плотность загрузки имеющихся ресурсов. В силу этого предлагается следующий вид функции цели:
$$
\frac
{t_{TA} 
\sum_{i=1}^{K} 
C_i
}
{
\sum_{j=1}^{N} 
t_j c_j}
t_{TA} \rightarrow \min
$$
\end{document}
